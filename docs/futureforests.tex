\documentclass[11pt,letter]{article}
\usepackage[top=0.70in, bottom=0.7in, left=1.1in, right=1.1in]{geometry}
\usepackage{graphicx} % Required for inserting images
\usepackage{xcolor} 
\definecolor{Accent}{HTML}{bd2b00} 
\usepackage[numbers,compress,super]{natbib}
\usepackage{gensymb}
\usepackage{hyperref}
\hypersetup{colorlinks,citecolor = Accent, linkcolor = Accent,urlcolor = Accent, breaklinks=true}
\usepackage{cleveref}
\usepackage[labelfont=bf]{caption}
\bibliographystyle{unsrtnat}

\RequirePackage[labelfont={bf,sf},%
                font={small, sf}]{caption}

\usepackage{lineno}
\renewcommand\linenumberfont{\normalfont\tiny\color{gray}}

\begin{document}

\title{Overlooked model uncertainties may misinform forest management strategies
% Alternative title: Integrating full model uncertainties could improve forest management
}

\author{Victor, Jérôme, Isabelle} % and more?
\date{}
\maketitle 
%emw19May -- overall this seems much better! The text is still not as tight as I would hope for a short format journal though. 

\noindent\rule{\textwidth}{0.3pt}
%emw19May -- One thing that is tricky here is the link the management. I think it's fine (and even good!) but it makes the number of links you need to make for the reader to get to your point longer. (For example, in your abstract you could just say 'Forests play a major role in mitigating climate change and maintaining that requires robust forecasts of their future composition ...' but the management connection takes more steps). Just being aware of this could help you make it as easy as possible on the reader. 
%emw19May --  alternative abstract text: 
Forests play a major role in mitigating climate change, but they must be carefully managed to maintain this role through increasing anthropogenic threats. Robust management requires robust forecasts, but current projections are the outcome of layered models and highly variable. Identifying the drivers of this variability---whether it is from different emissions scenarios, uncertainty in climate models or ecological models---is thus critical to advancing management. To address this, we compare over 1350 ecological models and climate scenarios in forecasts for forests across Europe. Our approach considers a gradient of more mechanistic (`process-based') to correlative models of species distributions. We find that difference between ecological models represent the largest source of uncertainty (40 to 64\%), surpassing both climate models and vastly different climate scenarios (e.g., SSP2 vs. SSP5). We also find areas with relatively consistent projections where management could take immediate action. Our results point to current limitations in ecological forecasting methods that make management difficult. At the same time they provide a framework to identify regions with the most consistent projections and those regions where high ecological uncertainty, which may benefit from diversified and more risk-adverse strategies until ecological forecasting advances.

% \textbf{Abstract---still need work:} Forests play a major role in mitigating climate change, but increasing threats to forests from climate change have heightened the importance of managing these systems. Robust forecasts of forest composition with increasing climate change are critical to this aim, but are currently highly variable. To help guide management in the face of this variability and understand where we can most rapidly reduce uncertainty through improved models, we compare over 1350 ecological models and climate scenarios in forecasts for forests across Europe. Our approach considers a gradient of more mechanistic (`process-based') to correlative models of species distributions and find that difference between ecological models represent the largest source of uncertainty (40 to 64\%), surpassing both climate models and vastly different climate scenarios (e.g., SSP2 vs. SSP5). We also find areas with relatively consistent projections [give overview of these and say that this could reduce uncertainty in how to manage for these areas]. Our results point to current limitations in ecological forecasting methods, identifies avenues to lower uncertainty % vvdm: too vague?
% and, suggests that managers need to internalize ecological uncertainty through diversified and more risk-adverse strategies.

\noindent\rule{\textwidth}{0.3pt}

\linenumbers

\subsection*{Main}

Forests are key to climate change mitigation policies and achieving carbon neutrality  \citep{Korosuo2023, Hyyrynen2023}. Yet, forests are increasingly under pressure. In Europe, temperatures are rising twice as fast as the global average \citep{CCCS2024}, and unprecedented pulses of tree mortality  have been reported in the last decade \citep{Senf2020}. As a result, some European forests are becoming net CO$_2$ sources \citep{Hadden2016, Karelin2021}, due to decreased growth \citep{Hadden2016, Woude2023}, larger burned areas \citep{Carnicer2022, Kelly2024}, and increased pest- and drought-induced dieback \citep{Karelin2021, Cienciala2024, Latifovic2024}. 

Forest managers working to minimize current threats while also promoting long-term adaptation to climate change rely on forecasts of shifts in critical tree species. Species shifts are predicted to have major impacts on timber production and on the forest economic sector \citep{Wessely2024, Hanewinkel2013}. To preserve the socio-economic functions of forests, managers need to know whether the current species will be able to tolerate future climate conditions, whether they can rely on natural regeneration, or whether they should consider new species opportunities. To date, however, ecological models have struggled to provide practical insights for forest management.

Different models, ranging from correlative to more mechanistic approaches, often provide highly divergent projections \citep{Morin2009, Keenan2011a, Cheaib2012, Takolander2019}. While it remains unclear under which conditions one approach is more reliable than another \citep{VanderMeersch2024}, most projections still rely on a limited set of models \citep{Dyderski2018, Wessely2024, Hanewinkel2013, Schueler2014}, masking significant uncertainties and ultimately increasing the risk of policy and management failures \citep{Dawson2011}. 

Gaining a better understanding of where uncertainties originate and how they relate is a first step to incorporate the current reality of uncertain forecasts into decision making \citep{Urban2016, Saltelli2020, Johnson2024, Simmonds2024}. % and to identify opportunities  to address policy-relevant questions ?
Current projections have several layers of climatological and biological uncertainties, including socioeconomic scenarios, global climate models, ecological models, down to the species level. If the main driver of variation across projections is the differences between ecological models, even more than different global emissions scenarios, it becomes critical to encompass a wide range of models. Failing to do so could lead to overly confident predictions about which species will or will not be able to survive in future climates, ultimately leading to counterproductive or even detrimental forest management decisions.
Consistency between projections can also reveal regions where models agree and where uncertainty is lower.
%emw19May -- I would cut the below sentence and see if you can get by without it. 
% We must incorporate the diversity of models and merge across ecological and climatological models to provide a complete picture of both the threats and opportunities for forests. 

%emw19May -- below, consider integrating methods more? Something like ... (you could also add 'hybrid' term here):
To this aim, we combined over 1350 projections from models of future forest tree species distributions (from 1970 to 2100, at a 0.1\degree~spatial resolution). We incorporated a wide range of ecological models, from more mechanistic (‘process-based’) to correlative models to `hybrid' models. To forecast from these models we also considered climate model variability (5 global climate models with different climate sensitivities) and two emissions (`forcing') scenarios, resulting in 10 different future climate simulations. Our dataset included 9 tree species, both deciduous and coniferous, adapted to diverse climatic conditions across Europe and representing two-thirds of Europe's forested area (Supp. Mat.).

Including variability from the ecological, climatological and emissions scenarios allowed us to quantify the contribution of each component to the total variation across projections. This approach represents a significant advancement over previous studies, which overlooked large portions of uncertainty. Such advances could lead to more informed decision-making to improve the resilience of forests. 

\begin{figure}
	\centering
	\includegraphics[width=1\linewidth]{figures/anova_within_species_byecoregion-1.pdf}
	\caption{\textbf{Ecological models represent the main source of uncertainty in future projections of species climatic suitability across European biomes (2080-2100).} This figure illustrates the level of uncertainty associated with projected changes in suitability relative to the historical period (1970–2000), \textbf{(a)} for each species within biome and \textbf{(b)} across all species and biomes. The 5 main European biomes are represented in \textbf{(c)}. We distinguish 5 main uncertainty sources: (i) the future scenario (SSP), (ii) the climate model used to generate the climate projections (GCM), (iii) the interactions between the SPP and the GCM, (iv) the species distribution modeling method (SDM), and (v) the interactions between SDM and climate projections (both GCMs and SSPs). For each species, the black line represents the mean projection, across all GCMs, SSPs, and SDMs. 90\% uncertainty ranges were calculated additively and symmetrically around the mean.}
	\label{fig:anovaspecies}  %emw19May -- what is (c)? Add to caption.
\end{figure}

\subsection*{Results and discussion}

Differences between ecological models consistently explained more variation than vastly different climate trajectories. Overall, we found that the choice of the ecological model explained 51\% of the variation across species and biomes, while uncertainty in the socio-economic pathway explained only 35\% of the variations (\Cref{fig:anovaspecies}). The current driest biome, the Mediterranean region, showed a consistent decline in suitability across all species. The Atlantic and Continental biomes displayed contrasting results depending on the species. 
The total uncertainty, however, makes few of these detected trends easy to act on (few were significant, \Cref{fig:anovaspecies}). One striking example is the climatic suitability change of sessile oak in the Atlantic region, where this species represents an important cultural and economic value, and for which more than 80\% of the uncertainty in climate change impact projections was due to variations among ecological models. 

% The differences among climate model projections (GCMs) and socioeconomic scenarios (SSPs). 
% in the Continental ecoregion, differences across models account for 73.7% of the total uncertainty of beech future suitability, despite being within the core of its present distribution. Similarly, SDM is the main source of uncertainty for sessile and pedunculate oaks (respectively 57% and 76.8%, Figure 22).

\subsubsection*{Mechanistic to correlative models contribute to high uncertainty...}

Accounting for more diverse ecological models led to a more comprehensive range of potential future change in suitability (\Cref{fig:cascade}), revealing the high divergence between ecological models and the persistent gaps in our understanding of species responses to climate change. Considering only correlative models would have misled to an overestimation of the contribution of climate projections (forcing scenarios, climate models, and their two-way interaction) to the total projection uncertainty in all regions, except the Mediterranean. In particular, divergence between climate models would have appeared to contribute as much as ecological models to projection uncertainty (on average, 36.6\% and 37.5\%, respectively). 

\begin{figure}
	\centering
	\includegraphics[width=0.75\linewidth]{figures/allspecies_cascade.pdf}
	\caption{\textbf{Considering a broad range of models provides a more comprehensive view of possible future scenarios.} This figure illustrates the average change in suitability across all species and all biomes, with the hirarchical contribution of each source of uncertainty. The top of each cascade represents the overall average change in suitability for each SSP. The level below shows the ensemble mean for each climate model (5 branches per SSP, corresponding to the 5 GCMs considered here), averaged across multiple ecological models. The next level represents the contribution of each ecological modeling approach (mechanistic, hybrid, and correlative, 3 branches per GCM). The final level displays the variations within each approach (e.g., different parameter calibrations or statistical algorithms), although for mechanistic models only a single calibration was available. This figure was inspired by the Figure 1.15 in IPCC, 2021: Chapter 1.}
	\label{fig:cascade}
\end{figure}

Using a comprehensive set of models allows to avoid the specific biases inherent to some modeling approaches. Our results revealed that models calibrated using current species range data (correlative and hybrid) consistently predict greater extinctions than models calibrated using experimental data (\Cref{fig:diffproj}). 
Current distribution data may capture only a portion of the climatic niche of a species, underestimating the range of conditions where it could survive \citep{Chevalier2024, NoguesBravo2016}.
These discrepancies between models can significantly alter country-level projections, and impact national strategies derived from them. 
For example, by the end of the 21st century, beech showed an average suitability decrease of -0.19 ($\pm$0.14) across its historical distribution when considering only models entirely calibrated with current species distribution data, leading to an average loss of 30.5\% of its historical distribution (\Cref{fig:diffproj}). But this decreasing trend vanishes once a broad range of models is accounted for (-0.028 $\pm$0.17).
Relying on a narrow set of models---especially derived from the same calibration process---undermines the robustness of projections \citep{Dawson2011}, and may ultimately bias decisions towards intensive intervention strategies (e.g. introduction of species outside their native range), potentially overlooking alternative strategies.
% Multi-model ensembles have been so far mostly restricted to statistical models (Simmonds et al, 2024), but we show here that there is a strong interest in considering a broader range of models to better characterize projections uncertainty. 

\begin{figure}
	\centering
	\includegraphics[width=1\linewidth]{figures/fagus_sankey.pdf}
	\caption{\textbf{Models build on current species range data project higher extinctions.} This figure illustrates the average projection of beech distribution under scenario SSP2-4.5, for each of the 3 ecological modeling approaches considered here: \textbf{(a)} correlative, \textbf{(b)} hybrid and \textbf{(c)} mechanistic models. The upper maps show the projected change in beech distribution for 2080–2100, relative to its historical distribution (1970-2000). The lower Sankey diagrams illustrate the temporal evolution of beech distribution, from its historical distributions to projected distributions for 2040–2060 and 2080–2100.}
	\label{fig:diffproj}
\end{figure}

\subsubsection*{Moving forward with management despite uncertainties}

Despite large uncertainties, comparing diverse models improve prediction robustness and enable to identify areas with relatively consistent projections where precise actions for adaptation can be decided with larger confidence (\Cref{fig:manag}).
Around the Mediterranean Basin, the models consistently predict less favorable climatic conditions for the species we considered here. In areas where most species are threatened, forest managers may thus consider introducing more drought-tolerant species. Along the Atlantic margin, the suitability of most species is also projected to decrease, except for the two Mediterranean species---pubescent and evergreen oaks--- which could replace less adapted temperate species such as beech \citep{Penuelas2003}. Mechanistic model projections are less pessimistic for deciduous oaks and beech (\Cref{fig:diffproj}), suggesting that some better-adapted populations could survive if the existing standing genetic variation is maintained and promoted by forest managers \citep{Brang2014}. Additionally, adapting management practices, such as decreasing stand density to limit competition for water, could support their long-term survival to drought events \citep{Young2023}.
Finally, boreal biomes in Scandinavian and Baltic countries are projected to get an overall increase of climatic suitability (\Cref{fig:manag}). These include Finland and Sweden, two very important forestry countries in terms of wood stock, added value, and forest-based workforce [cite].
Forests in these regions are dominated by two conifers species, Scots pine and spruce, favoured by commercial forest management. Insufficient experimental data prevented us from making mechanistic model projections for spruce, but uncertainty for Scot pine future suitability was very high. Models consistently project that temperate deciduous trees will become more competitive at the northern margin of their range (\Cref{fig:diffproj}), and the extending growing season could offer an opportunity to convert pure coniferous stands into mixed forest to increase their resilience \citep{Schauer2023}.

Our framework also allow to identify regions of high uncertainty, highlighting where diversification strategies are most required.
A large part of the Continental biome exhibit less clear trends (\Cref{fig:manag}), as well as mountainous regions at the transition between Mediterranean and Continental/Atlantic climates (Pyrenees, Massif Central, Balkans). A key lever of action in this region is the diversification of tree species, as well as increasing genetic diversity within populations, to mitigate the risks associated with uncertain future conditions \citep{Morin2014, Ammer2019, Pretzsch2021, Vospernik2024}. Promoting uneven-aged stands could also enhance forest stability by improving structural complexity and buffering against climate extremes \citep{Vangi2024, Zhang2024a}.

Even in these highly uncertain regions our approach still highlights some smaller zones of consistency. Models agree on a lower suitability for Scots pines (with uncertainty driven more by climatic models and scenarios, 45.7\%, than by ecological models, 30.8\%), which is the dominant species in more than 30\% of forests and a commercially very important species in several countries of Central Europe [cite]. %(such as Poland, Eastern Germany, Czech Republic, Belarus).
Projections also suggest that temperate deciduous species (e.g. beech, deciduous oaks) will be less affected by climate change, despite high uncertainties due to high divergence among ecological models (between 45.7 and 75.4\% of the total uncertainty). Together, these findings contribute to a more complete picture of both the threats and opportunities facing forests in Europe, and highlight regions where policies have to be carefully tailored to effectively address all uncertainties.


% supp idea: Forest management decisions can only be made alongside the development of economic sectors for deciduous species

\begin{figure}
	\centering
	\includegraphics[width=1\linewidth]{figures/suit3-1.pdf}
	\caption{\textbf{Accounting for uncertainties supports evidence-based forest management.} Our framework allows to identify 3 areas that differ in terms of uncertainty levels, future climate risks and levers of action to address them. The map shows the average suitability change across all climate models, all ecological models and all species. Dashed areas highlight regions where the three types of ecological models (mechanistic, hybrid and correlative) do not agree on the sign of suitability change.}
	\label{fig:manag}
\end{figure}
% conifer-dominated forestry

\subsubsection*{Advancing on two fronts: move forward with uncertainty while aiming to reduce it through improved ecological models}

The large uncertainties we found clearly raise questions about the robustness of existing projections, and highlight that further advances are necessary to provide the most useful projections for managers and for policy. 
Correlative models are becoming more mechanistic by integrating experimental data \citep{Wagner2023}---yet, with some caution \citep{Chevalier2024a}. Combining deep learning with physics-based models offers a promising approach to improve climate predictions \citep{Kochkov2024}, with a lower computational cost that could facilitate a more comprehensive estimation of uncertainties. These breakthroughs are continually enhancing our ability to make robust ecological and climatological forecasts, and could in turn reveal new innovative pathways for forest management. But the need for this improvement should not prevent us from providing a better framework to identify uncertainty in projections that can guide management, such the framework we offer here.  

The implications of these results extend beyond European forests. Mechanistic models have also been developed for North America and Asia \citep{Morin2007, Fang2022}. Their projections could be systematically integrated in a comprehensive framework such as ours, alongside correlative model projections. Such continental-scale uncertainty assessment---going beyond regional analyses \citep{Iverson2016}---would provide more robust guidance for forest management. This is particularly critical in countries where forests are managed at a broader scale than in Europe--such as the United States---and where it could thus be easier to incorporate uncertainty into large-scale decision making. 

Given the rapid pace of climate change, rather than debating over which modeling approach to favor, efforts should focus on bridging diverse methodologies to generate practical insights and support evidence-based decision making. Ultimately, as scientists, we need to be transparent about projection uncertainties if we expect forest managers to acknowledge and integrate them \citep{Saltelli2020}. This is how ecological modeling can become truly relevant to decision making in the face of a changing climate.


\subsection*{Methods}

\subsubsection*{Species distribution models}

We sought to encompass a broad diversity of species distribution models by including three different approaches: correlative models, mechanistic models and hybrid models (i.e. mechastic models calibrated like correlative models\citep{VanderMeersch2023}).

For the correlative approach, we selected four well-established models\citep{Valavi2022}: GLM with lasso regularization, GAM, BRT, and down-sampled Random Forest. For the mechanistic approach, we used the process-based model PHENOFIT. The model has been validated for several North American and European species, either in historical or Holocene climatic conditions \citep{Morin2007, Saltre2013, Duputie2015, Gauzere2020, VanderMeersch2024}. For the hybrid approach, we calibrated PHENOFIT using the same species occurrence data as correlative models\citep{VanderMeersch2023}. We optimized the parameters of the model using the covariance matrix adaptation evolution strategy\citep{Hansen2001}.

\subsubsection*{Climate projections}

Future simulations were run with the last Coupled Model Intercomparison Project Phase 6 (CMIP6) climate  projections, for 5 global climate models (GCMs) and 2 shared socio-economic pathways (SSPs). 
We used model projections that were downscaled to a 0.1° resolution with a statistical trend-preserving method (the cumulative distribution function transform), using the ERA5-Land reanalysis as a reference observational dataset between 1981 and 2010 \citep{Noel2022}. The five GCMs were GFDL-ESM4 \citep{Dunne2020}, IPSL-CM6A-LR \citep{Lurton2020}, MPI-ESM1-2-HR \citep{Mueller2018}, MRI-ESM2-0 \citep{Yukimoto2019} and UKESM1-0-LL \citep{Sellar2020}. They are considered as good representatives of the full CMIP6 ensemble \citep{Noel2022}. 

\subsubsection*{Uncertainty partitioning}

Our approach was inspired by the partitioning of uncertainties in climate projections initially developed by Hawkins and Sutton \citep{Hawkins2009, Hawkins2011}, which was subsequently enhanced with additional methodologies \citep{Yip2011, Lafferty2023}. Rather than using a simple variance decomposition approach, we perform an ANOVA-based variance decomposition to also estimate the importance of the two-way interaction effects. All analyses were performed in R \citep{RCT2024}.

Across all species, we partitioned three sources of uncertainty: the climate projection uncertainty related to the different GCMs, SSPs, and their interaction, the species distribution modeling uncertainty related to the differents SDMs. We also considered the interactions between SDMs and climate projections (GCMs and SSPs). For each year, the suitability of a cell was considered as a 21-year moving average suitability (e.g. 2040-2060 for the year 2050). We then computed the difference of suitability with the historical suitability (c1970-2000). For each GCM and each SSP, when multiple SDM projections were simulated within the same SDM approach (e.g. multiple algorithms for correlative approach), we kept one ensemble per approach. For each year $t$, we then applied a linear ANOVA to calculate the sums of squares attributable to each uncertainty source:
$$
{SS}_{tot} = {SS}_{GCM} + {SS}_{SSP} + {SS}_{GCM:SSP} + {SS}_{SDM} + {SS}_{SDM:GCM} + {SS}_{SDM:SSP} + {SS}_{residuals}
$$

We then computed 90\% uncertainty ranges additively and symmetrically around the mean projection (across all GCMs, SSPs, SDMs), e.g. for SDM uncertainty: $\pm1.645*\sigma*\frac{{SS}_{SDM}}{{SS}_{tot}}$.


\clearpage

\bibliography{phd_bibliography}

\end{document}
