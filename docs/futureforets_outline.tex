\documentclass[11pt,letter]{article}
\usepackage[top=0.20in, bottom=0.50in, left=1in, right=1in]{geometry}
\usepackage{graphicx} % Required for inserting images

\title{Overlooked uncertainties...}

\author{Victor, Isabelle, and more?}
\date{Feb 2025}

\begin{document}

\maketitle

\section*{Outline}

\subsection*{Introduction}

\begin{enumerate}

\item the baby: with climate change, we need to safeguard and adapt forests, as they play a major role in mitigating its effects

\item the werewolf: We have many models, we don't know which one is the best. Most decisions rely on limited models without much insight into what drives differences in projections. This can mask significant uncertainties and ultimately threaten the success of forest management decisions

\item the silver bullet: accept a greater diversity of models (scientific approaches, hypothesis), and figure out how to safeguard forests by gaining a better understanding of uncertainties, i.e., merging across biological and climatological components (uncertainty budget framework)


\end{enumerate}


\subsection*{Results, discussion}

\begin{enumerate}
	
\item models, methodology = significant and often overlooked source of uncertainty, even greater than variability of different climate projections

% \item niche models systematically bias projections towards more pessimistic scenarios

\item which implications in terms  of forest management? 
\begin{enumerate}
\item on average, uncertain projections = more possibilities to act? more adaptation measures
\item but high uncertainties may lead to \emph{laissez-faire}
\item we want to avoid that, how to translate uncertainties into decision-making?\\
$\rightarrow$ favor forest adaptation strategies resilient to a wide range of possible future conditions.
\item forest managers, policy makers:\\
rethink the way species distribution modeling is applied to forest management\\
\end{enumerate}

\item looking ahead: a call to action for the scientific community:
\begin{enumerate}
	\item substantial progress required to develop more reliable projections
	\item proper evaluation of the transferability?
	\item integration of fewer, but more robust models that incorporate mechanistic understanding? simple models can also be great!
	\item going further into uncertainty evaluation? (parameter uncertainty is totally ignored in process-explicit models...)
	$\rightarrow$ us, as scientists, we need to expose these uncertainties if we want forest managers to tackle them (transparency), i.e; build on similar framework to actually guide adaptation
\end{enumerate}








\end{enumerate}


\end{document}
