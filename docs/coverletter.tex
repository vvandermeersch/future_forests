\documentclass[11pt,a4paper]{article}
\usepackage[top=1.00in, bottom=1.0in, left=1.1in, right=1.1in]{geometry}
\usepackage{graphicx}
\usepackage{hyperref}
\usepackage[style=authoryear, backend=biber, natbib = true, style=numeric-comp, sortcites, hyperref, sorting=none, giveninits=true, doi=false,isbn=false,url=false,eprint=false, date=year, maxbibnames = 2, backend=bibtex]{biblatex}
\AtEveryBibitem{\clearfield{pages}}
\AtEveryBibitem{\clearlist{language}}
\AtEveryBibitem{%
  \clearfield{volume}%
  \clearfield{number}}
\AtEveryBibitem{\clearfield{title}}
\renewbibmacro{in:}{}
\usepackage[export]{adjustbox}


\addbibresource{phd_bibliography.bib} 

\begin{document}

\pagenumbering{gobble}

\noindent \includegraphics[width=0.5\textwidth, right]{/home/victor/Documents/Papiers officiels/Signatures/forestry_letterhead.png}

\vspace{0.75cm}

\noindent Dear Dr. Wake, 

\vspace{0.25cm} %emwAug10: Lots of edits, take or leave what works, and I am happy to read a final draft. 

\noindent Please consider our paper, entitled ``Overlooked model uncertainties may misinform forest management strategies'', for publication in \emph{Nature Climate Change}. 
This manuscript leverages methods from climatology to decompose ecological forecasts to advance at once ecological modeling and guide forest management amidst uncertainty.

\vspace{0.3cm}

\noindent Forests are already showing signs of becoming carbon sources, putting pressure on ecological forecasts to guide management now. Yet, current studies often rely on a narrow set of models and ignore a large part of the uncertainties\supercite{Wessely2024, Hanewinkel2013}, limiting their practical insights for forest management. Misinformed decision-making---because of incomplete forecasts---could have long-term adversarial consequences on ecosystems\supercite{Dawson2011, Urban2016}, driving forest declines and reduced carbon storage.

\vspace{0.3cm} 
\noindent Here, we apply methods from climatology designed to aid decision-making given many models and related uncertainties. We considered over 1,350 projections of tree species range shifts, with diverse ecological and climatological models, and across different emissions scenarios. By fully encompassing the different sources of uncertainty, we were able to quantify each source contribution to the forecast uncertainty, across species and across biomes.  %emwAug10: especially nice sentence here!
\vspace{0.3cm}

%emwAug10: don't bury something interested in () -- remember that () means I can just NOT read it if I want. 
\noindent Our results show that ecological models represent the largest source of uncertainty---up to two thirds---even under vastly different emission scenarios. While previous studies have strongly underestimated the overall forecast uncertainty, our workflow provides a comprehensive view of potential futures for forests. 
% Uncertainties vary between biomes and species, highlighting pathways to improve forest management. 
 %emwAug10: especially nice sentence below!
We identify regions where management could take immediate action with low risk of failure, and regions where models strongly disagree for which uncertainty management and diversification of options would be necessary.

\vspace{0.3cm}
%emwAug10: maybe loop back in here something about climatology if easy (though feels less critical). 
%emwAug10: Also, I would be careful to talk too much about Europe alone, I would angle more towards: we show how these methods work using Europe as a case study and/or mention 'in Europe and beyond' 
\noindent Our findings can reshape the debate on ecological forecasting of future ranges. While ecologists have spent decades debating types of ecological models, our results provide a path to guide forest policy and management while the science advances. 
% contribute to a more complete assessment of both the threats and opportunities facing forests in Europe, and highlight regions where policies have to be carefully tailored to effectively address all uncertainties. 
 %emwAug10: especially nice sentence below!
We advocate for a more systematic incorporation of uncertainties into decision-making\supercite{Dawson2011, Urban2016, Saltelli2020}, and suggest that managers need to internalize ecological uncertainty through diversified and more risk-adverse strategies.

\vspace{0.3cm}

\noindent All authors contributed to this work and approve this version for submission. The manuscript is 2200 words, with 59 references and 4 figures, and is not under consideration elsewhere. We hope you find it suitable for publication in \emph{Nature Climate Change}, and look forward to hearing from you. 

\vspace{0.5cm}
\noindent Sincerely, \\
\vspace{-1cm}\\
\hspace*{-0.5cm}
\includegraphics[scale=.5]{/home/victor/Documents/Papiers officiels/Signatures/signature_longue.png} \\
\vspace{-2cm}\\
\noindent Victor Van der Meersch, PhD\\
\noindent \emph{Forest \& Conservation Sciences}\\
\noindent \emph{University of British Columbia}

\clearpage

\paragraph{References}
\printbibliography[heading=none]


\newpage

\end{document}


